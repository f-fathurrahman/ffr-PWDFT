\documentclass[a4paper,11pt]{extarticle}
\usepackage[a4paper]{geometry}
\geometry{verbose,tmargin=2cm,bmargin=2cm,lmargin=2cm,rmargin=2cm}

%\usepackage{fontspec}
%\defaultfontfeatures{Ligatures=TeX}
%\setmainfont{Linux Libertine O}
%\setmonofont{Fira Mono}

\setlength{\parindent}{0cm}

\usepackage{hyperref}
\usepackage{url}
\usepackage{xcolor}

\usepackage{minted}
%\newminted{julia}{breaklines,fontsize=\footnotesize}
\newminted{fortran}{breaklines}

\definecolor{mintedbg}{rgb}{0.95,0.95,0.95}
\usepackage{mdframed}

\BeforeBeginEnvironment{minted}{\begin{mdframed}[backgroundcolor=mintedbg]}
\AfterEndEnvironment{minted}{\end{mdframed}}

\begin{document}

\title{User Guide for {\ttfamily ffr-PWDFT}}
\author{Fadjar Fathurrahman}
\date{}
\maketitle

\tableofcontents

\section{Introduction}

Welcome to {\tt ffr-PWDFT} documentation.

{\tt ffr-PWDFT} is a poor man's program (or collection of subroutines, as of now)
to carry out electronic structure calculations based on density functional theory
and plane wave basis set.

\section{Theory}

Kohn-Sham equations

Plane wave basis set

Pseudopotential

\section{Implementation}

Description of unit cell and atomic structures

Description of plane wave basis set: G-vectors and real space basis set

\begin{fortrancode}
PROGRAM testprog
  IMPLICIT NONE
END PROGRAM
\end{fortrancode}


\end{document}
